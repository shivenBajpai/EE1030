\documentclass[journal]{IEEEtran}
\usepackage[a5paper, margin=10mm]{geometry}
%\usepackage{lmodern} % Ensure lmodern is loaded for pdflatex
\usepackage{tfrupee} % Include tfrupee package

%iffalse
\let\negmedspace\undefined
\let\negthickspace\undefined
\usepackage{gvv-book}
\usepackage{gvv}
\usepackage{cite}
\usepackage{amsmath,amssymb,amsfonts,amsthm}
\usepackage{algorithmic}
\usepackage{graphicx}
\usepackage{textcomp}
\usepackage{xcolor}
\usepackage{txfonts}
\usepackage{listings}
\usepackage{enumitem}
\usepackage{mathtools}
\usepackage{gensymb}
\usepackage{comment}
\usepackage[breaklinks=true]{hyperref}
\usepackage{tkz-euclide} 
\usepackage{listings}                                        
%\def\inputGnumericTable{}                                 
\usepackage[latin1]{inputenc}                                
\usepackage{color}                                            
\usepackage{array}                                            
\usepackage{longtable}                                       
\usepackage{calc}                                             
\usepackage{multirow}                                         
\usepackage{hhline}                                           
\usepackage{ifthen}                                           
\usepackage{lscape}
\usepackage{tabularx}
\usepackage{array}
\usepackage{float}
\usepackage{multicol}

\newcommand{\BEQA}{\begin{eqnarray}}
\newcommand{\EEQA}{\end{eqnarray}}
%\newcommand{\define}{\stackrel{\triangle}{=}}

\setlength{\headheight}{1cm} % Set the height of the header box
\setlength{\headsep}{0mm}     % Set the distance between the header box and the top of the text


%\usepackage[a5paper, top=10mm, bottom=10mm, left=10mm, right=10mm]{geometry}


\setlength{\intextsep}{10pt} % Space between text and floats

% Marks the beginning of the document
\begin{document}
\onecolumn
\bibliographystyle{IEEEtran}
\vspace{3cm}

%\renewcommand{\theequation}{\theenumi}
\numberwithin{equation}{enumi}
\numberwithin{figure}{enumi}
\renewcommand{\thefigure}{\theenumi}
\renewcommand{\thetable}{\theenumi}

\title{Mains - 11.A}
\author{ai24btech11030 - Shiven Bajpai}
\maketitle

\section*{Section - A}
\begin{enumerate}
\item{ 
    Let $f\brak{x}= \begin{cases}
	   (x-1)^2 \sin\frac{1}{x-1}\abs{x} & \text{if} \: x\ne1 \\
      -1 & \text{if} \: x = 1 
    \end{cases} $
    be a real-valued function. Then the set of points where f(x) is not differentiable is \ldots \hfill (1981 - 2 Marks)
}\\

\item{
    Let $f\brak{x}= \begin{cases}
    \frac{x^3+x^2-16x+20}{(x-2)^2}& \text{if } x \ne 2\\
    k& \text{if } x = 2
    \end{cases}$
    
    If f(x) is continuous for all x, then k= \ldots \hfill(1981 - 2 Marks)
}\\


\item{
    A discontinuous function $y=f(x)$ satisfying $x^2 + y^2 = 4$ is given by $f(x)=$ \ldots \hfill (1982 - 2 Marks)
}\\

\item {
    $\lim_{x\to1} (1-x) \tan\frac{\pi x}{2} = $ \ldots \hfill (1984 - 2 Marks)
}\\
    
\item {
    If $f(x)= \begin{cases}\sin x,& x \ne n\pi, n = 0, \pm1, \pm2, \pm3, \ldots \\
    2,& \text{otherwise}\end{cases}$\\
    and $g(x) = \begin{cases}x^2 + 1,& x \ne 0, 2\\
    4,& x=0\\
    5,& x=2\end{cases}$\\
    then $\lim_{x\to0} g\sbrak{f\brak{x}}$ is \ldots \hfill(1986 - 2 Marks)
}\\

\item {
    $\lim_{x\to-\infty}\sbrak{\frac{x^4 \sin \brak{\frac{1}{x}} + x^2}{\brak{1 + \abs{x}^3}}} =$ \ldots \hfill (1987 - 2 Marks)
}\\

\item {
    If $f(9)=9, f^{\prime}(9)=4, \text{then} \lim_{x\to9}\frac{\sqrt{f(x)}-3 }{\sqrt{x}-3}$ equals \ldots \hfill (1988 - 2 Marks)
}\\

\item{
    $ABC$ is an isosceles triangle inscribed in a circle of radius $r$. If $AB = AC$ and $h$ is the altitude from $A$ to $BC$ then the triangle $ABC$ has perimeter $P=2(\sqrt{2hr-h^2})+\sqrt{2hr})$ and area $A=$ \ldots , also $\lim_{h\to0} \frac{A}{P^3} = $ \ldots \hfill (1989 - 2 Marks)
}\\

\item {
    $\lim_{x\to\infty}\brak{\frac{x+6}{x +1}}^{x+4} =$ \ldots \hfill (1990 - 2 Marks)
}\\

\item{
    Let $f(x) = x\abs{x}$. The set of points where $f(x)$ is twice differentiable is \ldots \hfill (1992 - 2 Marks)
}\\

\item{
    Let $f(x)= \sbrak{x} \sin\brak{\frac{\pi}{\sbrak{x+1}}}$, where $\sbrak{}$ denotes the greatest integer function. The domain of $f$ is \ldots and the points of discontinuity of $f$ in the domain are \ldots \hfill (1996 - 2 Marks)
}\\

\item{
    $\lim_{x\to0} \brak{\frac{1+5x^2}{1+3x^2}}^{\frac{1}{x^2}} =$ \ldots \hfill (1996 - 1 Mark)
}\\

\item{
    Let $f(x)$ be a continuous function defined for $1 \leq x \leq 3$. If $f(x)$ takes rational values for all $x$ and $f\brak{2} = 10$, then $f\brak{1.5} = $ \ldots \hfill (1997 - 2 Marks)
}\\
    
\end{enumerate}

\section*{Section - B}
\begin{enumerate}
	\item{
 If $\lim_{x\to a}\sbrak{f\brak{x}g\brak{x}}$ exists then both $\lim_{x\to a}f(x)$ and $\lim_{x\to a}g(x)$ exist. \hfill (1981 - 2 Marks)
}
\end{enumerate}

\section*{Section - C}
\begin{enumerate}
	\item{
 If $f(x) = \sqrt{\frac{x - \sin x}{x + \cos^2 x}}$, then $\lim_{x \to \infty} f\brak{x}$ is \hfill (1979)

    \begin{multicols}{2}
    \begin{enumerate}
        \item{$0$}
        \item{$\infty$}
        \columnbreak
        \item{$1$}
        \item{none of these}
    \end{enumerate}
    \end{multicols}
 
}
\end{enumerate}
\end{document}

