%iffalse
\let\negmedspace\undefined
\let\negthickspace\undefined
\documentclass[journal,12pt,twocolumn]{IEEEtran}
\usepackage{cite}
\usepackage{amsmath,amssymb,amsfonts,amsthm}
\usepackage{algorithmic}
\usepackage{graphicx}
\usepackage{textcomp}
\usepackage{xcolor}
\usepackage{txfonts}
\usepackage{listings}
\usepackage{enumitem}
\usepackage{mathtools}
\usepackage{gensymb}
\usepackage{comment}
\usepackage[breaklinks=true]{hyperref}
\usepackage{tkz-euclide} 
\usepackage{listings}                                        
%\def\inputGnumericTable{}                                 
\usepackage[latin1]{inputenc}                                
\usepackage{color}                                            
\usepackage{array}                                            
\usepackage{longtable}                                       
\usepackage{calc}                                             
\usepackage{multirow}                                         
\usepackage{hhline}                                           
\usepackage{ifthen}                                           
\usepackage{lscape}
\usepackage{tabularx}
\usepackage{array}
\usepackage{float}
\usepackage{gvv}

\newtheorem{theorem}{Theorem}[section]
\newtheorem{problem}{Problem}
\newtheorem{proposition}{Proposition}[section]
\newtheorem{lemma}{Lemma}[section]
\newtheorem{corollary}[theorem]{Corollary}
\newtheorem{example}{Example}[section]
\newtheorem{definition}[problem]{Definition}
\newcommand{\BEQA}{\begin{eqnarray}}
\newcommand{\EEQA}{\end{eqnarray}}
\newcommand{\define}{\stackrel{\triangle}{=}}
\theoremstyle{remark}
\newtheorem{rem}{Remark}

% Marks the beginning of the document
\begin{document}
\bibliographystyle{IEEEtran}
\vspace{3cm}

\title{Mains.2.B.1-14}
\author{ai24btech11030 - Shiven Bajpai}
\maketitle
\newpage
\bigskip

\renewcommand{\thefigure}{\theenumi}
\renewcommand{\thetable}{\theenumi}

\section*{Section - B}

\begin{enumerate}
	\item{$z$ and $w$ are two non zero complex numbers such that $\abs{z} = \abs{w}$ and $Arg\brak{z} + Arg\brak{w} = \pi$ then $z$ equals \hspace*{\fill} \brak{2002}
		\\
		\\
		\begin{tabular}{l l l l}
			\brak{a} $\overline{\omega}$ & \brak{b} $-\overline{\omega}$ & \brak{c} $\omega$ & \brak{d} $-\omega$
		\end{tabular}
		\\
		}
		
	\item{If $\abs{z-4}<\abs{z-2}$, its solution is given by \hspace*{\fill} \brak{2002}
		\\
		\center
		\begin{tabular}{l l}
			\brak{a} $Re(z)>0$ & \brak{b} $Re(z)<0$ \\
			\brak{c} $Re(z)>3$ & \brak{d} $Re(z)>2$
		\end{tabular}
		\center}
		
	\item{The locus of the centre of a circle which touches the circle $\abs{z-z_1}=a$ and $\abs{z-z_2}=b$ externally \brak{z, z_1, z_2 \text{ are complex numbers}} will be \hspace*{\fill} \brak{2002}
		\\
		\center
		\begin{tabular}{l l}
			\brak{a} an ellipse & \brak{b} a hyperbola \\
			\brak{c} a circle & \brak{d} none of these
		\end{tabular}
		\center}
		
	\item{If $z$ and $w$ are two non-zero complex numbers such that $\abs{zw}=1$ and $Arg\brak{z} - Arg\brak{w} = \frac{\pi}{2}$ then $\overline{z}w$ is equal to 
		\\
		\hspace*{\fill} \brak{2003}
		\center
		\begin{tabular}{l l l l}
			\brak{a} $-\iota$ & \brak{b} $1$ & \brak{c} $-1$ & \brak{d} $\iota$
		\end{tabular}
		\center}

	\item{Let $Z_1$ and $Z_2$ be two roots of the equation $Z^2 + aZ + b = 0$, $Z$ being complex. Further assume that the origin, $Z_1$ and $Z_2$ form an equilateral triangle. Then \hspace*{\fill} \brak{2003}
		\\
		\center
		\begin{tabular}{l l}
			\brak{a} $a^2 = 4b$ & \brak{b} $a^2 = b$ \\
			\brak{c} $a^2 = 2b$ & \brak{d} $a^2 = 3b$
		\end{tabular}
		\center}

	\item{If $\brak{\frac{1-\iota}{1+\iota}}^x = 1$ then \hspace*{\fill} \brak{2003}
		\center
		\begin{tabular}{l}
			\brak{a} $x = 2n + 1$, where n is any positive integer \\
			\brak{b} $x = 4n$, where n is any positive integer \\ 
			\brak{c} $x = 2n$, where n is any positive integer \\
			\brak{d} $x = 4n + 1$, where n is any positive integer
		\end{tabular}
		\center}
		
	\item{Let $z$ and $w$ be complex numbers such that $\overline{z} + \iota\overline{w} = 0$ and $arg\brak{zw} = \pi$ then $arg\brak{z}$ equals \hspace*{\fill} \brak{2004}
		\\
		\center
		\begin{tabular}{l l l l}
			\brak{a} $\frac{5\pi}{4}$ & \brak{b} $\frac{\pi}{2}$ & \brak{c} $\frac{3\pi}{4}$ & \brak{d} $\frac{\pi}{4}$
		\end{tabular}
		\center}

	\item{If $z=x-\iota y$ and $z^{\frac{1}{3}}=p+\iota q$, then $\frac{\frac{x}{p} + \frac{y}{q}}{p^2 + q^2}$ is equal to 
		\\
		\hspace*{\fill} \brak{2004}
		\center
		\begin{tabular}{l l l l}
			\brak{a} $-2$ & \brak{b} $-1$ & \brak{c} $2$ & \brak{d} $1$
		\end{tabular}
		\center}

	\item{If $\abs{z^2 - 1} = \abs{z}^2 + 1$, then z lies on \hspace*{\fill} \brak{2004}
		\\
		\center
		\begin{tabular}{l l}
			\brak{a} an ellipse & \brak{b} the imaginary axis \\
			\brak{c} a circle & \brak{d} the real axis
		\end{tabular}
		\center}
	

	\item{If the cube roots of unity are 1, $\omega$, $\omega^2$ then the roots of the equation $\brak{x-1}^3 + 8 = 0$, are \hspace*{\fill} \brak{2004}
		\center
		\begin{tabular}{l}
			\brak{a} $-1,-1+2\omega,-1-2\omega ^2$ \\
			\brak{b} $-1,-1,-1$ \\
			\brak{c} $-1, 1-2\omega, 1-2\omega ^2$ \\
			\brak{d} $-1, 1+2\omega, 1+2\omega ^2$
		\end{tabular}
		\center}

	\item{If $z_1$ and $z_2$ are two non-complex numbers such that $\abs{z_1 + z_2} = \abs{z_1} + \abs{z_2}$, then arg\brak{z_1} - arg\brak{z_2} is equal to\hspace*{\fill} \brak{2005}
		\\
		\center
		\begin{tabular}{l l l l}
			\brak{a} $\frac{\pi}{2}$ & \brak{b} $-\pi$ & \brak{c} $0$ & \brak{d} $\frac{\pi}{2}$
		\end{tabular}
		\center}

	\item{If $\omega = \frac{z}{z-\frac{1}{3}\iota}$ and $\abs{\omega} = 1$, then z lies on \hspace*{\fill} \brak{2005}
		\\
		\center
		\begin{tabular}{l l}
			\brak{a} an ellipse & \brak{b} a circle \\
			\brak{c} a straight line & \brak{d} a parabola
		\end{tabular}
		\center}

	\item{The value of $\sum_{k=1}^{10}\brak{sin\brak{\frac{2k\pi}{11}}+\iota cos\brak{\frac{2k\pi}{11}}}$ is 
		\\
		\hspace*{\fill} \brak{2006}
		\center
		\begin{tabular}{l l l l}
			\brak{a} $\iota$ & \brak{b} $1$ & \brak{c} $-\iota$ & \brak{d} $-1$
		\end{tabular}
		\center}

	\item{If $z^2 + z + 1 = 0$, where z is a complex number, then the value of $$\brak{z+\frac{1}{z}}^2 + \brak{z^2+\frac{1}{z^2}}^2 +\brak{z^3+ \frac{1}{z^3}}^2 + \dots + \brak{z^6+\frac{1}{z^6}}^2 $$ is \hspace*{\fill} \brak{2006}
		\\
		\center
		\begin{tabular}{l l l l}
			\brak{a} $18$ & \brak{b} $54$ & \brak{c} $6$ & \brak{d} $12$
		\end{tabular}
		\center}

\end{enumerate}
\end{document}
