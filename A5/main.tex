\documentclass[journal]{IEEEtran}
\usepackage[a5paper, margin=10mm]{geometry}
%\usepackage{lmodern} % Ensure lmodern is loaded for pdflatex
\usepackage{tfrupee} % Include tfrupee package

%iffalse
\let\negmedspace\undefined
\let\negthickspace\undefined
\usepackage{gvv-book}
\usepackage{gvv}
\usepackage{cite}
\usepackage{amsmath,amssymb,amsfonts,amsthm}
\usepackage{algorithmic}
\usepackage{graphicx}
\usepackage{textcomp}
\usepackage{xcolor}
\usepackage{txfonts}
\usepackage{listings}
\usepackage{enumitem}
\usepackage{mathtools}
\usepackage{gensymb}
\usepackage{comment}
\usepackage[breaklinks=true]{hyperref}
\usepackage{tkz-euclide} 
\usepackage{listings}                                        
%\def\inputGnumericTable{}                                 
\usepackage[latin1]{inputenc}                                
\usepackage{color}                                            
\usepackage{array}                                            
\usepackage{longtable}                                       
\usepackage{calc}                                             
\usepackage{multirow}                                         
\usepackage{hhline}                                           
\usepackage{ifthen}                                           
\usepackage{lscape}
\usepackage{tabularx}
\usepackage{array}
\usepackage{float}
\usepackage{multicol}

\newcommand{\BEQA}{\begin{eqnarray}}
\newcommand{\EEQA}{\end{eqnarray}}
%\newcommand{\define}{\stackrel{\triangle}{=}}

\setlength{\headheight}{1cm} % Set the height of the header box
\setlength{\headsep}{0mm}     % Set the distance between the header box and the top of the text


%\usepackage[a5paper, top=10mm, bottom=10mm, left=10mm, right=10mm]{geometry}


\setlength{\intextsep}{10pt} % Space between text and floats

% Marks the beginning of the document
\begin{document}
\onecolumn
\bibliographystyle{IEEEtran}
\vspace{3cm}

%\renewcommand{\theequation}{\theenumi}
\numberwithin{equation}{enumi}
\numberwithin{figure}{enumi}
\renewcommand{\thefigure}{\theenumi}
\renewcommand{\thetable}{\theenumi}

\title{JEEM - 6Sep2020 - Shift1 - 16-30}
\author{ai24btech11030 - Shiven Bajpai}
\maketitle

\iffalse
\begin{multicols}{4}
\begin{enumerate}
    \item 
    \item 
    \item 
    \item 
\end{enumerate}
\end{multicols}
\fi

\begin{enumerate}
    \item $\lim_{x \rightarrow 1} \brak{\frac{\int_{0}^{\brak{x-1}^2} t \cos \brak{t^2} dt}{\brak{x-1}\sin\brak{x-1}}}$ \hfill (6 Sep 2020 - S1)
    
    \begin{multicols}{4}
    \begin{enumerate}
        \item is equal to $1$
        \item is equal to $\frac{1}{2}$
        \item does not exist
        \item is equal to $\frac{-1}{2}$
    \end{enumerate}
    \end{multicols}
    
    \item If $\sum_{i=1}^{n}(x_i - a) = n$ and $\sum_{i=1}^{n} (x_i - a)^2 = na$, $(n,a >1)$ then the standard deviation of $n$ observations $x_1, x_2, x_3\ldots x_n$ is: \hfill (6 Sep 2020 - S1)
    
    \begin{multicols}{4}
    \begin{enumerate}
        \item $nv(a-1)$
        \item $v(na-1)$
        \item $a-1$
        \item $v(a-1)$
    \end{enumerate}
    \end{multicols}
    
    \item If $\alpha$ and $\beta$ be two roots of the equation $x^2 -64x+256 = 0$. Then the value of $(\frac{\alpha^3}{\beta^5})^\frac{1}{8}+(\frac{\beta^3}{\alpha^5})^\frac{1}{8}$ is: \hfill (6 Sep 2020 - S1)

    \begin{multicols}{4}
    \begin{enumerate}
        \item $1$
        \item $3$
        \item $2$
        \item $4$
    \end{enumerate}
    \end{multicols}
    
    \item The position of a moving car at time $t$ is given by $f(t) = at^2+bt+c, t > 0$, where $a$, $b$ and $c$ are real numbers greater than 1. Then the average speed of the car over the time interval $\sbrak{t_1, t_2}$ is attained at the point \hfill (6 Sep 2020 - S1)

    \begin{multicols}{4}
    \begin{enumerate}
        \item $\frac{(t_1+t_2)}{2}$
        \item $2a(t_1+t_2)+b$
        \item $\frac{(t_2-t_1)}{2}$
        \item $a(t_2-t_1)+b$
    \end{enumerate}
    \end{multicols}
    
    \item If $I_1 = \int_{0}^{1}(1-x^{50})^{100} dx$ and $I_2 = \int_{0}^{1}(1-x^{50})^{101} dx$ such that $I_2 = \alpha I_1$ then $\alpha$ equals to \hfill (6 Sep 2020 - S1)
    
    \begin{multicols}{4}
    \begin{enumerate}
        \item $\frac{5050}{5049}$
        \item $\frac{5050}{5051}$
        \item $\frac{5051}{5050}$
        \item $\frac{5049}{5050}$
    \end{enumerate}
    \end{multicols}
    
    \item If $\vec{a}$ and $\vec{b}$ are unit vectors, then the greatest value of $\sqrt{3} \abs{\vec{a} + \vec{b}} + \abs{\vec{a} - \vec{b}}$ is \hfill (6 Sep 2020 - S1) \\
    
    \item Let $AD$ and $BC$ be two vertical poles at $A$ and $B$ respectively on a horizontal ground. If $AD = 8\text{m}$, $BC = 11\text{m}$ and $AB = 10\text{m}$; then the distance (in meters) of a point $M$ on $AB$ from the point $A$ such that $MD^2+MC^2$ is minimum is \hfill (6 Sep 2020 - S1)
    
    \item Let $f : \mathcal{R} \rightarrow \mathcal{R}$ be defined as $f(x)= \begin{cases}x^5\sin\brak{\frac{1}{x}} + 5x^2,& x < 0 \\
    0,& x = 0\\
    x^5\cos\brak{\frac{1}{x}} + \lambda x^2,& x > 0\end{cases}$\\
    
    The value of $\lambda$ for which $f^{\prime\prime}(0)$ exists, is \hfill (6 Sep 2020 - S1) \\ 
    
    \item The angle of elevation of the top of a hill from a point on the horizontal plane passing through the foot of the hill is found to be $45^\degree$. After walking a distance of $80$ meters towards the top, up a slope inclined at an angle of $30^\degree$ to the horizontal plane, the angle of elevation of the top of the hill becomes $75^\degree$. Then the height of the hill (in meters) is \hfill (6 Sep 2020 - S1) \\
    
    \item Set $A$ has $m$ elements and set $B$ has $n$ elements. If the total number of subsets of $A$ is $112$ more than the total number of subsets of $B$, then the value of $m\times n$ is \hfill (6 Sep 2020 - S1)
\end{enumerate}

\end{document}

