%iffalse
\let\negmedspace\undefined
\let\negthickspace\undefined
\documentclass[journal,12pt,twocolumn]{IEEEtran}
\usepackage{cite}
\usepackage{amsmath,amssymb,amsfonts,amsthm}
\usepackage{algorithmic}
\usepackage{graphicx}
\usepackage{textcomp}
\usepackage{xcolor}
\usepackage{txfonts}
\usepackage{listings}
\usepackage{enumitem}
\usepackage{mathtools}
\usepackage{gensymb}
\usepackage{comment}
\usepackage[breaklinks=true]{hyperref}
\usepackage{tkz-euclide} 
\usepackage{listings}                                        
%\def\inputGnumericTable{}                                 
\usepackage[latin1]{inputenc}                                
\usepackage{color}                                            
\usepackage{array}                                            
\usepackage{longtable}                                       
\usepackage{calc}                                             
\usepackage{multirow}                                         
\usepackage{hhline}                                           
\usepackage{ifthen}                                           
\usepackage{lscape}
\usepackage{tabularx}
\usepackage{array}
\usepackage{float}
\usepackage{gvv}
\usepackage{multicol}

\newtheorem{theorem}{Theorem}[section]
\newtheorem{problem}{Problem}
\newtheorem{proposition}{Proposition}[section]
\newtheorem{lemma}{Lemma}[section]
\newtheorem{corollary}[theorem]{Corollary}
\newtheorem{example}{Example}[section]
\newtheorem{definition}[problem]{Definition}
\newcommand{\BEQA}{\begin{eqnarray}}
\newcommand{\EEQA}{\end{eqnarray}}
\newcommand{\define}{\stackrel{\triangle}{=}}
\theoremstyle{remark}
\newtheorem{rem}{Remark}

% Marks the beginning of the document
\begin{document}
\bibliographystyle{IEEEtran}
\vspace{3cm}

\title{Mains - 14.A+B}
\author{ai24btech11030 - Shiven Bajpai}
\maketitle
\newpage
\bigskip

\renewcommand{\thefigure}{\theenumi}
\renewcommand{\thetable}{\theenumi}

\section*{Section - E}
\begin{enumerate}
	\item{
			Prove that $\cos \tan^{-1} \sin \cot^{-1} x = \sqrt{\frac{x^2 + 1}{x^2 + 2}}$. \hfill (2002 - 5 Marks)
		}
\end{enumerate}

\onecolumn
\section*{Section - F}
\begin{enumerate}
	\item{
		Match The Following \hfill (2005 - 6M)
		\begin{multicols}{2}
			\textbf{Column I}\\
			\begin{enumerate}
				\item{$\sum_{i=1}{\infty}\tan^{-1}\brak{\frac{1}{2i^2}}=t$, then tan t =}\\
				\item{Sides $a,b,c$ of a triangle ABC are in AP and $\cos\theta_1=\frac{a}{b+c}, \cos\theta_2=\frac{b}{a+c}, \cos\theta_3=\frac{c}{a+b}$ then $tan^2\brak{\frac{\theta_1}{2}}+\tan^2\brak{\frac{\theta_3}{2}} = $}\\
				\item{A line is perpendicular to $x+2y+2z=0$ and passes through (0,1,0). The perpendicular distance of this line from the origin is}
			\end{enumerate}
			\columnbreak
			\textbf{Column II}\\
			\begin{enumerate}
				\item{1}\\
				\item{$\frac{\sqrt{5}}{3}$}\\
				\item{$\frac{2}{3}$}
			\end{enumerate}
		\end{multicols}}
	\item{
		Let $(x,y)$ be such that $\sin^{-1}\brak{ax}+\cos^{-1}\brak{bxy}=\frac{\pi}{2}.\\$
		Match the statements in Column 1 with statements in Column II and indicate your answer by darkening the appropriate bubble in the 4x4 matrix given in the ORS.
		\begin{multicols}{2}
			\begin{enumerate}
				\item{If $a=1$ and $b=0$, then $(x, y)$}
				\item{If $a=1$ and $b=1$, then $(x, y)$}
				\item{If $a=1$ and $b=2$, then $(x, y)$}
				\item{If $a=2$ and $b=2$, then $(x, y)$}
			\end{enumerate}
			\columnbreak
			\begin{enumerate}
				\item{lies on the circle $x^2 + y^2 = 1$}
				\item{lies on $(x^2-1)(y^2-1)=0$}
				\item{lies on $y=x$}
				\item{lies on $(4x^2-1)(y^2-1)=0$}
			\end{enumerate}
		\end{multicols}}
	
	\hrule
		\text{}\\
		\textbf{DIRECTIONS(Q.3): }\textit{Following questions has matching lists. The codes for the lists have choices (a), (b), (c) and (d) out of which ONLY ONE is correct.}\\
	\hrule

	\item{
		\begin{multicols}{2}
			\begin{enumerate}
				\item{$\brak{\frac{1}{y^2}\brak{\frac{\cos\brak{\tan^{-1}y}+y\sin\brak{\tan^{-1}y}}{\cot\brak{\sin^{-1}y}+\tan\brak{\sin^{-1}y}}}^2 + y^4}^\frac{1}{2}$} takes value\\
				\item{If $\cos x+\cos y+\cos z = 0 = \sin x+\sin y+\sin z$ then possible value of $\cos\frac{x-y}{2} is$}\\
				\item{If $\cos\brak{\frac{\pi}{4}-x} \cos 2x+\sin x\sin 2x\sec x=\cos x\sin 2x\sec x+\cos\brak{\frac{\pi}{4}+x}\cos 2x$ then possible value of $\sec x$ is}\\
				\item{If $\cot\brak{\sin^{-1}\sqrt{1-x^2}}=\sin\brak{\tan^{-1}\brak{x\sqrt{6}}},x\neq 0$}
			\end{enumerate}
			\columnbreak
			\begin{enumerate}
				\item{$\frac{1}{2}\sqrt{\frac{5}{3}}$}\\
				\item{$\sqrt{2}$}\\
				\item{$\frac{1}{2}$}\\
				\item{1}
			\end{enumerate}
		\end{multicols}
		\textbf{Codes:}\\
		\\
		\begin{tabular}{c c c c c}
			& \textbf{P} & \textbf{Q} & \textbf{R} & \textbf{S} \\
			\brak{a} & 4 & 3 & 1 & 2 \\
			\brak{b} & 4 & 3 & 2 & 1 \\
			\brak{c} & 3 & 4 & 2 & 1 \\
			\brak{d} & 3 & 4 & 1 & 2 \\
		\end{tabular}
		}
\end{enumerate}

\twocolumn
\section*{I - Integer Value Correct Type}
\begin{enumerate}
	\item{
		The number of real solutions of the equation $$\sin ^{-1}\brak{\sum_{i=1}^{\infty} x^{i+1}-x \sum_{i=1}^{\infty} \brak{\frac{x}{2}}^i} = \frac{\pi}{2} - \cos^{-1}\brak{\sum_{i=1}^{\infty}\brak{\frac{-x}{2}}^i-\sum_{i=1}^{\infty} \brak{-x}^i}$$ lying in the interval $\brak{-\frac{1}{2},\frac{1}{2}}$ is. (Here, the inverse trignometric function $\sin^{-1}x$ and $\cos^{-1}x$ assume values in $[-\frac{\pi}{2}, \frac{pi}{2}]$ and $[0, \pi]$ respectively \hfill (JEE Adv. 2018)
}
\\
\item{
		The value of $$\sec^{-1}\brak{\frac{1}{4}\sum_{k=0}^{10} \sec\brak{\frac{7\pi}{10}+\frac{k\pi}{10} \sec{\frac{7\pi}{12}+\frac{\brak{k+1}\pi}{2}}}}$$ in the interval $[-\frac{\pi}{4},\frac{3\pi}{4}]$ equals \_\_\_\_. \hfill (JEE Adv 2019)	
	}
\end{enumerate}

\twocolumn
\section*{Section B - JEE Main / AIEEE}
\begin{enumerate}
	\item{
		$\cos^{-1}\brak{\sqrt{\cos\alpha}}-tan^{-1}\brak{\sqrt{\cos\alpha}}$ , then $\sin x =$ \hfill [2002]
		\begin{multicols}{2}
		\begin{enumerate}
			\item{$\tan^2 \brak{\frac{\alpha}{2}}$}
			\item{$\cot^2 \brak{\frac{\alpha}{2}}$}
			\columnbreak
			\item{$\tan\alpha$}
			\item{$\cot \brak{\frac{\alpha}{2}}$}
		\end{enumerate}
		\end{multicols}
	}
	\item{
			The trignometric equation $sin^{-1} x = 2 sin^{-1}a$ has a solution for \hfill [2003]
		\begin{multicols}{2}
		\begin{enumerate}
			\item{$\abs{\alpha}\geq\frac{1}{\sqrt{2}}$}
			\item{$\frac{1}{2} < \abs{\alpha} < \frac{1}{\sqrt{2}}$}
			\columnbreak
			\item{all real values of a}
			\item{$\abs{\alpha} < \frac{1}{2}$}
		\end{enumerate}
		\end{multicols}
	}
	\item{
			If $\cos^{-1}x - \cos^{-1}\frac{y}{2} = \alpha$, then $4x^2 - 4xy \cos \alpha + y^2$ is equal to \hfill [2005]
		\begin{multicols}{2}
		\begin{enumerate}
			\item{$2 \sin 2\alpha$}
			\item{$4 $}
			\columnbreak
			\item{$4 \sin^2 \alpha $}
			\item{$-4 \sin^2 \alpha $}
		\end{enumerate}
		\end{multicols}
	}
	\item{
			If $\sin^{-1} \brak{\frac{x}{5}} + \cosec^{-1}\brak{\frac{5}{4}} = \frac{\pi}{2}$, then the value of x is \hfill [2007]
		\begin{multicols}{2}
		\begin{enumerate}
			\item{$4$}
			\item{$5$}
			\columnbreak
			\item{$1$}
			\item{$3$}
		\end{enumerate}
		\end{multicols}
	}
	\item{
			The value of $\cot\brak{\cosec^{-1}\frac{5}{3} + \tan^{-1}\brak{2}{3}}$
		\begin{multicols}{2}
		\begin{enumerate}
			\item{$\frac{6}{17}$}
			\item{$\frac{3}{17} $}
			\columnbreak
			\item{$ \frac{4}{17}$}
			\item{$ \frac{5}{17}$}
		\end{enumerate}
		\end{multicols}
	}
	\item{
			If x,y,z are in AP and $\tan^{-1}x, \tan^{-1}y and \tan^-1{z}$ are also in A.P, then \hfill [JEE M 2013]
		\begin{multicols}{2}
		\begin{enumerate}
			\item{$x=y=z$}
			\item{$2x=3y=6z$}
			\columnbreak
			\item{$6x=3y=2z$}
			\item{$6x=4y=3z$}
		\end{enumerate}
		\end{multicols}
	}
	\item{
			Let $\tan^{-1}y = \tan^{-1}x + \tan^{-1}\brak{\frac{2x}{1-x^2}}$, where $\abs{x} < \frac{1}{\sqrt{3}}$ Then a value of y is \hfill [JEE M 2015]
		\begin{multicols}{2}
		\begin{enumerate}
			\item{$\frac{3x - x^3}{1 + 3x}$}
			\item{$\frac{3x + x^3}{1 + 3x} $}
			\columnbreak
			\item{$\frac{3x - x^3}{1 - 3x} $}
			\item{$\frac{3x + x^3}{1 - 3x} $}
		\end{enumerate}
		\end{multicols}
	}
	\item{
			If $\cos^{-1}\brak{\frac{2}{3x}} + \cos^{-1}\brak{\frac{3}{4x}} = \frac{\pi}{2} \brak{x > \frac{3}{4}}$, then x is equal to \hfill [JEE M 2019 - 9 Jan M]
		\begin{multicols}{2}
		\begin{enumerate}
			\item{$\frac{\sqrt{145}}{12}$}
			\item{$\frac{\sqrt{145}}{10} $}
			\columnbreak
			\item{$\frac{\sqrt{146}}{12} $}
			\item{$\frac{\sqrt{145}}{11} $}
		\end{enumerate}
		\end{multicols}
	}
\end{enumerate}

\end{document}
