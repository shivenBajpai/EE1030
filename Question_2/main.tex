\documentclass[journal]{IEEEtran}
\usepackage[a5paper, margin=10mm]{geometry}
%\usepackage{lmodern} % Ensure lmodern is loaded for pdflatex
\usepackage{tfrupee} % Include tfrupee package

%iffalse
\let\negmedspace\undefined
\let\negthickspace\undefined
\usepackage{gvv-book}
\usepackage{gvv}
\usepackage{cite}
\usepackage{amsmath,amssymb,amsfonts,amsthm}
\usepackage{algorithmic}
\usepackage{graphicx}
\usepackage{textcomp}
\usepackage{xcolor}
\usepackage{txfonts}
\usepackage{listings}
\usepackage{enumitem}
\usepackage{mathtools}
\usepackage{gensymb}
\usepackage{comment}
\usepackage[breaklinks=true]{hyperref}
\usepackage{tkz-euclide} 
\usepackage{listings}                                        
%\def\inputGnumericTable{}                                 
\usepackage[latin1]{inputenc}                                
\usepackage{color}                                            
\usepackage{array}                                            
\usepackage{longtable}                                       
\usepackage{calc}                                             
\usepackage{multirow}                                         
\usepackage{hhline}                                           
\usepackage{ifthen}                                           
\usepackage{lscape}
\usepackage{tabularx}
\usepackage{array}
\usepackage{float}
\usepackage{multicol}

\newcommand{\BEQA}{\begin{eqnarray}}
\newcommand{\EEQA}{\end{eqnarray}}
%\newcommand{\define}{\stackrel{\triangle}{=}}

\setlength{\headheight}{1cm} % Set the height of the header box
\setlength{\headsep}{0mm}     % Set the distance between the header box and the top of the text


%\usepackage[a5paper, top=10mm, bottom=10mm, left=10mm, right=10mm]{geometry}


\setlength{\intextsep}{10pt} % Space between text and floats

% Marks the beginning of the document
\begin{document}
\onecolumn
\bibliographystyle{IEEEtran}
\vspace{3cm}

%\renewcommand{\theequation}{\theenumi}
\numberwithin{equation}{enumi}
\numberwithin{figure}{enumi}
\renewcommand{\thefigure}{\theenumi}
\renewcommand{\thetable}{\theenumi}

\title{Question 1-1.5-31}
\author{ai24btech11030 - Shiven Bajpai}
\maketitle

\renewcommand{\thefigure}{\theenumi}
\renewcommand{\thetable}{\theenumi}

\textbf{Question: } Find the coordinates of a point $P$, which lies on the line segment joining the points $A \brak{-2, 2}$ and $B \brak{2, -4}$ such that $AP = \frac{3}{7}AB$.	

\textbf{Solution: } $P$ divides $AB$ in ratio $3:4$ i.e $1:\frac{4}{3}$
\begin{align}
	P &= \frac{kA + B}{k + 1} \\
	&= \frac{\frac{4}{3}\myvec{-2 \\ 2} + \myvec{2 \\ -4}}{\frac{4}{3} + 1} \\
	&= \frac{4\myvec{-2 \\ 2} + 3\myvec{2 \\ -4}}{4 + 3} \\
	&= \frac{1}{7}\myvec{-2 \\ -4}
\end{align}

\begin{figure}[H]
	\centering
	\includegraphics[width=0.75\columnwidth]{Figures/Figure.png}
	\caption{A plot of all points}
	\label{fig}
\end{figure}

Code for this plot
\begin{lstlisting}
	Codes/section.py
\end{lstlisting}

\end{document}
