\documentclass[journal]{IEEEtran}
\usepackage[a5paper, margin=10mm]{geometry}
%\usepackage{lmodern} % Ensure lmodern is loaded for pdflatex
\usepackage{tfrupee} % Include tfrupee package

%iffalse
\let\negmedspace\undefined
\let\negthickspace\undefined
\usepackage{gvv-book}
\usepackage{gvv}
\usepackage{cite}
\usepackage{amsmath,amssymb,amsfonts,amsthm}
\usepackage{algorithmic}
\usepackage{graphicx}
\usepackage{textcomp}
\usepackage{xcolor}
\usepackage{txfonts}
\usepackage{listings}
\usepackage{enumitem}
\usepackage{mathtools}
\usepackage{gensymb}
\usepackage{comment}
\usepackage[breaklinks=true]{hyperref}
\usepackage{tkz-euclide} 
\usepackage{listings}                                        
%\def\inputGnumericTable{}                                 
\usepackage[latin1]{inputenc}                                
\usepackage{color}                                            
\usepackage{array}                                            
\usepackage{longtable}                                       
\usepackage{calc}                                             
\usepackage{multirow}                                         
\usepackage{hhline}                                           
\usepackage{ifthen}                                           
\usepackage{lscape}
\usepackage{tabularx}
\usepackage{array}
\usepackage{float}
\usepackage{multicol}

\newcommand{\BEQA}{\begin{eqnarray}}
\newcommand{\EEQA}{\end{eqnarray}}
%\newcommand{\define}{\stackrel{\triangle}{=}}

\setlength{\headheight}{1cm} % Set the height of the header box
\setlength{\headsep}{0mm}     % Set the distance between the header box and the top of the text


%\usepackage[a5paper, top=10mm, bottom=10mm, left=10mm, right=10mm]{geometry}


\setlength{\intextsep}{10pt} % Space between text and floats

% Marks the beginning of the document
\begin{document}
\onecolumn
\bibliographystyle{IEEEtran}
\vspace{3cm}

%\renewcommand{\theequation}{\theenumi}
\numberwithin{equation}{enumi}
\numberwithin{figure}{enumi}
\renewcommand{\thefigure}{\theenumi}
\renewcommand{\thetable}{\theenumi}

\title{JEEM - 8Jan2020 - Shift2 - 16-30}
\author{ai24btech11030 - Shiven Bajpai}
\maketitle

\begin{enumerate}
    \item The area (in sq. units) of the region ${\brak{x,y} \in \mathcal{R} : x^2 \leq y \leq 3 - 2x}$, is:
    \begin{multicols}{4}
    \begin{enumerate}
        \item $\frac{31}{3}$
        \item $\frac{32}{3}$
        \item $\frac{29}{3}$
        \item $\frac{34}{3}$
    \end{enumerate}
    \end{multicols}
    
    \item Let $S$ be the set of all functions $f:\sbrak{0,1} \rightarrow \mathcal{R}$, which are continuous on $\sbrak{0, 1}$ and differentiable on $\brak{0, 1}$. Then for every $f$ in $S$, there exists a $c \in \brak{0,1}$, depending on $f$, such that:
    \begin{multicols}{2}
        \begin{enumerate}
            \item $\frac{\brak{f\brak{1}-f\brak{c}}}{\brak{1-c}} = f^\prime\brak{c} $
            \item $|f\brak{c} - f\brak{1}| < |f^\prime\brak{c}| $
            \columnbreak
            \item $|f\brak{c} + f\brak{1}| < \brak{1 + c}|f ^\prime \brak{c}| $
            \item $|f\brak{c} - f\brak{1}| < \brak{1 - c}|f^\prime\brak{c}|$
        \end{enumerate}
    \end{multicols}

    \item The differential equation of the family of curves, $x^2 = 4b\brak{y + b}, b \in \mathcal{R}$, is:
    \begin{multicols}{2}
    \begin{enumerate}
        \item $xy^{\prime\prime} = y^\prime$
        \item $x\brak{y^\prime} 2 = x + 2yy^\prime$
        \columnbreak
        \item $x\brak{y ^\prime } 2 = x - 2yy^\prime$
        \item $x\brak{y ^\prime } 2 = 2yy ^\prime - x$
    \end{enumerate}
    \end{multicols}

    
    \item The system of linear equations
    \begin{align*}
        \lambda x + 2y + 2z &= 5 \\
        2\lambda x + 3y + 5z &= 8 \\
        4x + \lambda y + 6z &= 10     
    \end{align*}
    has:
    
    \begin{multicols}{2}
        \begin{enumerate}
            \item no solution when $\lambda = 2$
            \item infinitely many solutions when $\lambda = 2$
            \columnbreak
            \item no solution when $\lambda = 8$
            \item a unique solution when $\lambda = -8$
        \end{enumerate}
    \end{multicols}
    
    \item If the $10^\text{th}$ term of an A.P. is $\frac{1}{20}$ and its $20^\text{th}$ term is $\frac{1}{10}$ , then the sum of its first 200 terms is:
    \begin{multicols}{2}
        \begin{enumerate}
            \item $50 \frac{1}{4}$
            \item $100$
            \columnbreak
            \item $50$
            \item $100 \frac{1}{2}$
        \end{enumerate}
    \end{multicols}
    
    \item Let a line $y = mx \: \brak{m > 0}$ intersect the parabola, $y^2 = x$ at a point $P$, other than the origin. Let the tangent to it at $P$ meet the x-axis at the point $Q$. If $area\brak{\Delta OPQ} = 4 \text{ sq.units}$, then $m$ is equal to $\ldots$\\
    
    \item Let $f\brak{x}$ be a polynomial of degree 3 such that $f\brak{-1} = 10$, $f\brak{1} = -6$, $f\brak{x}$ has a critical point at $x = -1$ and $f^\prime\brak{x}$ has a critical point at $x = 1$. Then the local minima at $x$ = $\ldots$\\
    
    \item $\frac{\sqrt{2}\sin\alpha}{\sqrt{1 + \cos2\alpha}} = \frac{1}{7}$ and $\sqrt{\frac{1-\cos2\beta}{2}} = \frac{1}{\sqrt{10}}$, $\alpha, \beta \in \brak{0, \frac{\pi}{2}}$, then $\tan\brak{\alpha + 2\beta}$ is equal to $\ldots$\\
    
    \item The number of 4 letter words (with or without meaning) that can be made from the eleven letters of the word ``EXAMINATION'' is $\ldots$\\
    
    \item The sum, $\sum_{n=1}^{7}\frac{n\brak{n+1}\brak{2n+1}}{4}$ is equal to $\ldots$\\
\end{enumerate}

\end{document}

